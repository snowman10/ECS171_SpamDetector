\documentclass{article}
\usepackage{graphicx} % Required for inserting images

\title{ECS 171 Project Report}
\author{Team 7}
\date{May 23rd, 2023}

\begin{document}

\maketitle

\section{Introduction}
\subsection{Problem Statement}
Message spam is an ongoing problem in the 21st century as millions of malicious texts are sent each day. Machine learning methods could 
help us predict whether or not an incoming message should be flagged as spam depending on what the message contains. We chose the SMS 
dataset by UCI Machine Learning to use as our test subject as we form a model that can flag incoming spam messages as accurately as 
possible.
\subsection{Dataset Description}
Our dataset consists of 5,574 messages, some of which are spam and others which are ham (non-spain). Messages labeled spam are malicious 
messages, while ham are genuine ones. 3,375 SMS messages are randomly chosen ham messages from the NUS SMS Corpus (which is a dataset 
that contains over 10,000 real messages for research purposes). 425 SMS messages are manually taken from the Grumbletext website. 
Another 450 are from Caroline Tag’s PhD thesis. The last 1,324 messages are from SMS Spam Corpus v.0.1 Big (which has been used for 
research in the past).
\subsection{Goal}
The goal of this project is to use machine learning algorithms to detect whether a message is considered spam or not. This can be 
done by traversing the messages within the dataset and counting the number of spam related elements. From this project, we will be 
able to explore different machine learning algorithms and learn how to extract our own attributes from raw data.
\end{document}
